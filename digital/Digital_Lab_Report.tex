\documentclass[12pt]{article}
\usepackage[margin=1in]{geometry}
\usepackage{graphicx}

\begin{document}

\title {Analog and Digital Mixing of Signals}
\author {Marta Pastore}
\date {March 7, 2014}
\maketitle

\begin {abstract}
The goal of this lab is to understand the difference between analog and
digital mixing of two signals. Ideally, the result of the two should be
similar but we experienced trouble with the DC offset in the digital
case. Furthermore, in this lab we were able to set the tap coefficients
for a 5/8 FIR filter, which are listed in Table 2. These coefficients
produce a respose close to the expected response of the filter which is
shown on Figure 20.

\end {abstract}

\section {Introduction}
In this lab we were introduced to various advanced equipment used for
the mixing of signals and processing of these signals. More
specifically, we explored the functions of a field programmable gate
array or FPGA when inputing two SRS oscillator signals. The FPGA
contains a series of programmable logic blocks and can therefore perform
different tasks at once, each completed independently from one
another. The FPGA we used has a built in analog-to-digital converter or
an ADC card to which we connected the input signals. The connections to
the ADC card determines whether the signal is sampled or mixed. In
addition, we explored the workings of a finite impulse response filter
or FIR filter inside the FPGA. The way an FIR filter works is it takes a
time series of samples, uses these to implement a convolution,
multiplies each sample by some coefficient and adds the results
together. The coefficient values are calculated for a specific frequency
response of the FIR filter. With the understanding of the FPGA and FIR
filter, we were able to implement a Digital-Down-Converter (DDC) that
mixes and filters our input signal.

\section {Methods}
We began by exploring how to control the SRS Function Generators and the
Pulsar sampler from the command line. To do this the DFEC module was
imported in our python code to set the values for the SRS and
Pulsar. Within the module, the function set srs was used to control the
two SRS used.To control how to take samples, the sampler function was
used.
\subsection {Nyquist Sampling and Aliasing}
From Nyquist sampling it is known that one must sample at a rate that is
more than twice the maximum frequency in the sprectrum of the signal in
order to avoid aliasing. To
explore this idea we took samples from a single sine wave. We chose a
sample frequency, $\nu_{sampl}$, of 100kHz. The signal frequency,
$\nu_{sig}$, was set to be some fraction of $\nu_{sampl}$. The values of
the signal frequency for which data was obtained are listed in Table 1. 
\begin {table}[!h]
\center
\begin {tabular}{|c|c|}
\hline
$\nu_{sig}$       &        fx$\nu_{sampl}$  \\
\hline
10KHz             &        0.1$\nu_{sampl}$ \\
\hline
20kHz             &        0.2$\nu_{sampl}$ \\
\hline
30kHz             &        0.3$\nu_{sampl}$ \\
\hline
40kHz             &        0.4$\nu_{sampl}$ \\
\hline
50kHz             &        0.5$\nu_{sampl}$ \\
\hline
60kHz             &        0.6$\nu_{sampl}$ \\
\hline
70kHz             &        0.7$\nu_{sampl}$ \\
\hline
80kHz             &        0.8$\nu_{sampl}$ \\
\hline
90kHz             &        0.9$\nu_{sampl}$ \\
\end {tabular} 
\caption {The values of the signal fequency for which data was
  collected.}
\label {Table.1} 
\end {table}   
At each $\nu_{sig}$, N contiguous samples are taken and we set N equal
to 256. For each data set, we plotted the digitally sampled waveform
versus time and the Fourier power spectrum.

Next, we set $\nu_{sig}$ equal to $\nu_{sampl}$ which is 100kHz and took
256 samples. A plot of the sampled waveform versus time was obtained and
is shown in Figure 1. We then violated
the Nyquist's criterion by setting $\nu_{sig}$ equal to 29.9$\nu_{sampl}$ and
a plot of the samples waveform versus time was created and is shown in
Figure 2. 
\begin {figure}[h!]
\centering
\includegraphics[width = 120mm,scale = 1]
{figure_10.png}
\caption{\label{rvd} A plot of the sampled waveform versus time for $\nu_{sig}$ = $\nu_{sampl}$ = 100kHz.}
\end {figure}

\begin {figure}[h!]
\centering
\includegraphics[scale = 0.5]
{figure_11.png}
\caption{\label{rvd} A plot of the sampled waveform versus time for
  $\nu_{sig}$  = 29.9$\nu_{sampl}$ with $\nu_{sampl}$ = 100kHz.}
\end {figure}

\begin {figure}[h!]
\centering
\includegraphics[width = 120mm, scale = 1]
{leakage.png}
\caption{\label{rvd} A plot of the sampled waveform vs time and the
  power spectrum for N = 10,000.}
\end {figure}

Now we caculate the power spectrum for many more than N output
frequencies over the Nyquist range
$-\nu_{sampl}/2$ to $+\nu_{sampl}/2(1-2/N)$
This means we make the
frequency increment much smaller than
$\Delta\nu = \nu_{sampl}/N$
We chose N=10,000 in order to achieve
this and we set the $\nu_{sig}$ = 0.1$\nu_{sampl}$, where $\nu_{sampl}$
is 100kHz. We then plotted the sampled waveform vs time and the power
spectrum which is shown in Figure 3.


\subsection {Basic Double Sideband (DSB) Mixer Operation}
In this section we used two SRS synthesizer oscillators as inputs to a
mixer to explore the spectra and the waveform in the DSB mixing
process. The SRS synthesizer works up to 30MHz. We set one of the
synthesizers as our local oscillator with frequency $\nu_{lo}$ equal to
300kHz. The other synthesizer was set as a signal oscillator with frequencies
$\nu_{sig}$ = $\nu_{lo}$ $\pm$ $\delta$$\nu$. Here $\delta$$\nu$ is the
frequency difference and we set that equal to 5kHz. We sampled for
both the sum and differance frequencies using two types of mixers.

\subsubsection {The Mini-Circuits ZAD-1 Mixer}
We combined the two signals, $\nu_{lo}$ and $\nu_{sig}$, using a
Mini-Circuits ZAD-1 mixer, which has three ports. We connected
$\nu_{lo}$ and $\nu_{sig}$ to the two outer ports R and L. The middle
port, I, is the output with the mixed frequencies and this was
connected to a DC sampler. We plotted the sampled signal and the power
spectra versus frequency for both cases, $\nu_{sig}$ = $\nu_{lo}$ $\pm$ $\delta$$\nu$, and they are
shown in Figure 4 and 5. For the
$\nu_{sig}$ = $\nu_{lo}$ - $\delta$$\nu$ case, we took the Fourier
transform of the waveform and removed the sum frequency component by
zeroing both the real and imaginary portions. We then attempted to recreate the
signal from the filtered transform by taking the inverse fourier
transform. A plot of the sampled signal versus time was obtained.  
\begin {figure}[h!]
\centering
\includegraphics[width = 120mm, scale = 1]
{mixing2.png}
\caption{\label{rvd} A plot of the sampled signal, the power spectra,
  and the fourier transform of $\nu_{sig}$ = $\nu_{lo}$ + $\delta$$\nu$. }
\end {figure}
\begin {figure}[h!]
\centering
\includegraphics[width = 120mm, scale = 1]
{mixing1.png}
\caption{\label{rvd} A plot of the sampled signal, the power spectra,
  and the fourier transform of $\nu_{sig}$ = $\nu_{lo}$ - $\delta$$\nu$. }
\end {figure}
\subsubsection {Digital Mixing on an FPGA}
We digitally mixed the same input frequencies as for the Mini-Circuits
case, $\nu_{lo}$ =300kHz and $\nu_{sig}$ = 295kHz, by collecting data
using an FPGA hosted on the ROACH. The FPGA has an ADC card with various ports
but for this case we connected our two signals to ports 2 and 3 in order to obtain a
digitalized mixed signal. In this case the sample frequency was preset
to 200MHz. A plot of the power spectra versus frequency was obtained for
the mixed signal.

Next, instead of mixing two input signals digitally, we mixed a
single signal with a local oscillator that is derived from the sample
clock of the FPGA. Our signal frequency was set to 2MHz and we connected this
signal to port 4 on the ADC card. We used 
{\begin{verbatim} echo -ne "x00\x00\x00\x02" > lo_freq\end{verbatim}}
to set the local oscillator frequency. What we are doing here
is mixing our signal frequency once with a cosine wave and with a sine
wave of the chosen oscillator frequency. A plot of the resulting
waveform and power spectrum is shown in Figure 6 and 7.
\begin {figure}[h!]
\centering
\includegraphics[width = 120mm, scale = 1]
{lovecos.png}
\caption{\label{rvd} A plot of the waveform and power spectrum of a
  20MHz signal wave mixed with a cosine wave. }
\end {figure}
\begin {figure}[h!]
\centering
\includegraphics[width = 120mm, scale = 1]
{lovesine.png}
\caption{\label{rvd} A plot of the waveform and power spectrum of a
  20MHz signal wave mixed with a sine wave. }
\end {figure}

\subsection {Setting FIR filter coefficient}
We are now including eight registers for inputting the coefficients for
the FIR filter circuit in the FPGA. Each coefficient has two registers
one for the real part, $coeff_{real}$, and one for the imaginary part,
$coeff_{imag}$. The coefficients were chosen as to implement a 5/8
bandpass filter. This means that when we divide the band into 8 positive
and negative frequency channels, we would extract the five channels
centered around zero. The coefficient values chosen are shown in Table
2. 

\begin {table}[!h]
\center
\begin {tabular}{|c|c|}
\hline
Frequency(MHz)       &        Coefficient  \\
\hline
-100                 &        0.1250000 \\
\hline
-75                  &        -0.0517767 \\
\hline
-50                  &        -0.01250000 \\
\hline
-25                  &        0.03017767 \\
\hline
0                    &        0.6250000 \\
\hline
25                   &        0.3017767 \\
\hline
50                   &        -0.1250000 \\
\hline
75                   &        -0.0517767\\
\end {tabular} 
\caption {The coefficient values for the 5/8 bandpass FIR filter.}
\label {Table.2} 
\end {table}  

\section {Results}
\subsection {Effects of the Nyquist criterion}
We digitally sampled at a frequency of 100kHz and took data for the
values of $\nu_{sig}$ shown on Table 1. In this case the Nyquist
frequency is half of the sampling frequency or 50kHz. At this frequency
we are sampling every half period; therefore we only expect the peaks
and troughs of the signal to be sampled. The resulting plot we obtained
at the Nyquist frequency is shown in Figure 8 and it agrees with what we
expect from sampling at the Nyquist frequency.
\begin {figure}[h!]
\centering
\includegraphics[width = 120mm, scale = 1]
{go5.png}
\caption{\label{rvd} A plot of the sampled waveform and power spectrum
  at $\nu_{sig}$ = 0.5$\nu_{sampl}$, Nyquist frequency. }
\end {figure}

For the values of $\nu_{sig}$ we are undersampling the
sinusoid and this allows there to be an alias for each $\nu_{sig}$. An alias is a
function that produces the same set of samples. For example when we plot
the sampled waveform and fourier transform at $\nu_{sig}$ =
0.1$\nu_{sampl}$ and $\nu_{sig}$ = 0.9$\nu_{sampl}$, shown in Figure 9
and 10, we can see that the peaks of the fourier transform in both cases
occur at 10kHz eventhough they are different signal
frequencies. This is known as aliasing and we experienced this also for
$\nu_{sig}$ = 0.2$\nu_{sampl}$ with $\nu_{sig}$ = 0.8$\nu_{sampl}$,
shown in Figure 11 and 12, which peak at 20kHz, for $\nu_{sig}$ =
0.7$\nu_{sampl}$ with $\nu_{sig}$ = 0.3$\nu_{sampl}$, shown in Figure 13
and 14,which peak at 30kHz and for $\nu_{sig}$ = 0.6$\nu_{sampl}$ with
$\nu_{sig}$ = 0.4$\nu_{sampl}$, shown in Figure 15 and 16, which peak at
40kHz.
It can be noted that the original $\nu_{sig}$ is fully recovered at the
frequencies below the Nyquist frequency but decreased at frequecies
above the Nyquist frequency. This is observed because the Nyquist
frequency is the highest frequency that can be coded at a given sampling
rate in order to be able to fully reconstruct the signal. 

In Figure 1 the plot for $\nu_{sig}$ = $\nu_{sampl}$ = 100kHz is shown
and we can see that the result is approximately a straight line close to
zero amplitude. This is because we are sampling once per period so the
data results in a straight line. 

In Figure 2 we explored the case of $\nu_{sig}$ = 29.9$\nu_{sampl}$, a
signal frequecy way below the Nyquist frequency. Here we determine a
frequency of 10kHz compared to a signal frequency of 2.99MHz. We are
taking so little samples per period that we are no where near our
original frequency. Once again an example of how the Nyquist frequency
is the highest frequency at which we can get back our original
frequency.

Furthermore, in Figure 3 we illustrate the sample frequency waveform and
the power spectrum of the case in which we set the number of samples, N,
equal to 10,000. This will make the frequency increments smaller. Making
the output frequencies closer together
gives a more continuous frequency coverage in the power spectrum. In the
power spectrum shown in figure 3, there are two peaks at around 10kHz
and -10kHz but there is a series of smaller power outputs on either side
of the two main peaks. Because an FFT has a finite length, there is a
window on the data that is taken. A window in the time domain results in
a convolution in the frequency domain with the transform of the
window. This transform is a Sinc function which has infinite
width. Therefore, the ripples of the Sinc function shown up a leakage
far from the spectral peaks that do not fit the FFT's length. 

Looking in Figure 3 at the frequency values of the highest peaks, we get
a difference between the two peaks of about 1.17kHz. The total time span
over which the samples were taken is 100,000$\mu$s. The inverse of this
value is 10Hz.

\begin {figure}[h!]
\centering
\includegraphics[width = 120mm, scale = 1]
{go1.png}
\caption{\label{rvd} A plot of the sampled waveform and power spectrum
  at $\nu_{sig}$ = 0.1$\nu_{sampl}$ }
\end {figure}
\begin {figure}[h!]
\centering
\includegraphics[width = 120mm, scale =1]
{go9.png}
\caption{\label{rvd} A plot of the sampled waveform and power spectrum
  at $\nu_{sig}$ = 0.9$\nu_{sampl}$. }
\end {figure}
\begin {figure}[h!]
\centering
\includegraphics[width = 120mm, scale =1]
{go2.png}
\caption{\label{rvd} A plot of the sampled waveform and power spectrum
  at $\nu_{sig}$ = 0.2$\nu_{sampl}$. }
\end {figure}
\begin {figure}[h!]
\centering
\includegraphics[width = 120mm, scale =1]
{go8.png}
\caption{\label{rvd} A plot of the sampled waveform and power spectrum
  at $\nu_{sig}$ = 0.8$\nu_{sampl}$. }
\end {figure}
\begin {figure}[h!]
\centering
\includegraphics[width = 120mm, scale =1]
{go3.png}
\caption{\label{rvd} A plot of the sampled waveform and power spectrum
  at $\nu_{sig}$ = 0.3$\nu_{sampl}$. }
\end {figure}
\begin {figure}[h!]
\centering
\includegraphics[width = 120mm, scale =1]
{go7.png}
\caption{\label{rvd} A plot of the sampled waveform and power spectrum
  at $\nu_{sig}$ = 0.7$\nu_{sampl}$. }
\end {figure}
\begin {figure}[h!]
\centering
\includegraphics[width = 120mm, scale =1]
{go6.png}
\caption{\label{rvd} A plot of the sampled waveform and power spectrum
  at $\nu_{sig}$ = 0.6$\nu_{sampl}$.}
\end {figure}
\begin {figure}[h!]
\centering
\includegraphics[width = 120mm, scale =1]
{go4.png}
\caption{\label{rvd} A plot of the sampled waveform and power spectrum
  at $\nu_{sig}$ = 0.4$\nu_{sampl}$, Nyquist frequency. }
\end {figure}

\subsection {Analog and Digital DSB Mixing Observations}
In Figures 4 and 5, the data for mixing with a Mini-Circuits ZAD-1
mixer for two cases, $\nu_{sig}$ =
$\nu_{lo}$ $\pm$ $\delta$$\nu$, is illustrated. For the addition case,
Figure 4, it can be seen that in both the fourier and power plots there
is a peak close to zero, which represents the difference between two
frequencies, +5kHz and -5KHz. There are also peaks at +605kHz and
-605kHz; these represent the addition of our two input
frequencies. In this case the lower sideband goes from -5kHz to -605kHz
and the upper sideband goes from +5kHz to +605kHz. Similarly for the
differance case, Figure 5, there are peaks at +5kHz and -5kHz as well as
+595kHz and -595kHz. Here the lower sideband ranges from -5kHz to
-595kHz and the upper sideband gors from +5kHz to +595kHz. 

Focusing now on the data from the difference case, Figure 4, the
waveform plotted matched the trace seen on the
oscilliscope. Furthermore, looking at the fourier transform plot, we
zeroed out frequencies that were higher than -5kKhz and +5kHz. This is
known as fourier filtering. When we took the inverse transform on this
new data we obtained the waveform shown in Figure 17. Here we have a
waveform of frequency 5kHz just like we wanted. We checked to see if
there were any imaginary components in the waveform and there were and
it is propably because we did not zero out all of the negative frequencies. 
\begin {figure}[h!]
\centering
\includegraphics[scale = 0.5]
{fourierfiltering.png}
\caption{\label{rvd} A plot of the fourier filtered waveform.}
\end {figure}

The result of mixing the two signals $\nu_{sig}$ = 295kHz and
$\nu_{lo}$ = 300kHz using digital DSD mixing with the FPGA is shown in
Figure 18. What we expected to see was a waveform of frequency 5kHz and
peaks in the power spectrum for the two input frequencies and 
the addition and the difference of the two frequencies like we
observed in the analog mixing case shown in Figure 5. Instead we only
observed peaks at around 300kHz. This means that our signal did not get
mixed properly. The reason for this result has to do with DC
offset. Since we are taking 2048 samples at 200MHz, we have a a
resolution of 100kHz. Our $\delta$$\nu$ is only 5kHz which is much
smaller than the resolution frequency. So we do not have enough
resolution to separate 5kHz and 0kHz, which is the DC frequency. Because the ROACH has
a DC offset, our 5kHz signal is being overlooked by this offset. Also,
this offset makes it so that when any other frequency is mixed with the DC
signal, the same frequency appears as the output. This explains why we
only see a peak at around 300kHz. 

When looking at Figures 6 and 7, these show our 2MHz signal mixed with a
cosine signal and a sine signal, repectively. Here the high peaks in the
power spectrum happen at around +2MHz and -2MHz which corresponds to our signal
frequency. We took the data for the cosine and sine mixing and combined
it to reconstruct the complex valued sinudoid that this digital
Single-Sideband (SSB) mixer outputs. The result of the combined power
spectrum and the waveform of the extracted sinusoid are shown in
Figure 19. The highest peak in the power spectrum occurs at approximately
1.5MHz and we chose to filter out this specific frequency. We did this
by setting the power to zero for all the other frequencies and the
waveform of this filtered frequency is shown in Figure 19. The frequency
of the plotted waveform is about 1.3MHz, which is really close to the
aimed 1.5MHz peak.

When comparing between SSB and DSB mixing we look at Figures 17 and
19. In Figure 17 we see the waveform of a 5kHz fourier filtered
signal. Similarly, in Figure 19 we see the waveform of a 1.5MHz fourier
filtered signal. This waveform was obtained usign a SSB mixer and it is
safe to say that it looks more like a sine wave than Figure 19, which was
obtained using a DSB mixer. It seems that SSB mixer is better at fourier
filtering a signal than the DSB mixer. Now focusing on Figures 18 and 19,
we can compare the power spectrum that results from DSB mixing and SSB
mixing, respectively. Here we can see that Figure 18 has two main peaks
that corresponds to the double sideband, while Figure 19 only has one
main peak that represents a single sideband. In addition, the power
spectrum of the SSB mixer, Figure 19, has a couple of tiny peaks. It
should result in just a single peak. This could be a sign that the
power spectrum is more realistic for the DSB than the SSB in our case,
since in the DSB we get exactly two peaks in the power spectrum. 

\begin {figure}[h!]
\centering
\includegraphics[width = 120mm, scale =1]
{roachpart1.png}
\caption{\label{rvd} A plot of the waveform and power spectrum that
  shows the results of digitally mixing  $\nu_{sig}$ = 295kHz and $\nu_{lo}$ = 300kHz with an FPGA.}
\end {figure}
\begin {figure}[h!]
\centering
\includegraphics[width = 120mm, scale =1]
{completessb.png}
\caption{\label{rvd} The first plot shows the power spectrum of the
  combined sine and cosine respose of a 2MHz signal.The second plot
  shows the waveform of the extracted 1.5MHz peak.}
\end {figure}

\subsection{FIR filter response}
The coefficients chosen for the FIR filter are shown in Table 2. The
expected response of our 5/8 bandpass filter is shown in Figure 20. Our
signal frequency for this case is 200MHz but since we know we must sample at
half the signal frequency, our filter only takes in frequencies between
100MHz and -100MHz to avoid aliasing. The response of our filter in time
domain is also shown in Figure 20 and here we see the coefficients with
respect to time. 

\begin {figure}[h!]
\centering
\includegraphics[width = 120mm, scale =1]
{firfilter.png}
\caption{\label{rvd} The first plot shows the expected response of a 5/8
bandpass filter on a 200MHz signal.}
\end {figure}

 

\section {Conclusion}
In this lab we mixed two signals using an analog and a digital mixer. We
plotted out data and found out that the power spectrum for these two
cases is not the same. For the digital mixing we did not see any signs
of mixing which was due to DC offset described earlier. In the future we
would like to fix this problem by paying more attention to the value of
the input frequency and making sure that the difference between the two
is closer to the resolution of the mixer. 

Furthermore, we will like to test our coefficients by programing them to
the FIR filter running on the FPGA and empirically characterize the
shape of our filter. 

\section {Acknowledgements}
I would like to thank my lab partner Kyle Moses for his contributions to
the results of this lab. I would also like to thank Karto Keatings and
Baylee Bordwell for helping with the obtaining and interpretation of my data.


\end {document}
  