\documentclass[12pt]{article}
\usepackage{graphicx}
\usepackage[margin=1in]{geometry}



\begin{document}

\title {Analog Lab}
\author {Marta Pastore}
\date {February 10, 2014}
\maketitle

\begin {abstract}

\end {abstract}

\section {Introduction}

\section {Methods}

\subsection {Impedance}

Before the main components of a circuit can be explored, the concept of
impedance,Z, needs to be well understood. Electrical impedance is
defined as the amount of opposition a circuit applies to any incoming
current when a voltage is applied. Impedance is therefore analogous to
friction in more physical terms. All of the different components in a
circuit have an impedance and therefore contribute to the overall
opposition of current in the circuit.

\subsection {Resistor}

A resistor is a two-terminal electrical component through which current
can flow. The impedance of a resistor, Zr, is equal to its resistance
R. Ohm's Law, V=IR is a very useful equation in understanding how a
resistor behaves. Furthermore, Resistors can be connected in series to
give a total resistance of Re  = R1 + R2 +...+ Rn. In this configuration
the voltage across the resistors is the same but the current is
different. Similarly resistor can be connected in parallel to give a
total resistance of 1/Re = 1/R1 + 1/R2 +...+ 1/Rn. In this configuration
the current through the resistors is the same but the voltage is
different.

Resistors can be used to build a voltage divider, which is a circuit
that gives an output voltage that is a fraction of the input
voltage. Two resistors connected in series can be considered a voltage
divider. To test this we took two 1500Ω resistors and connected them in
series and applied a 5V DC voltage source(Figure 1).
\begin {figure}[!h]
\centering 
\includegraphics[scale = 1.0]{resistive_voltage_divider.png}
\caption{\label{rvd} A resistive voltage divider with two 1500Ω 
resistors connectect two a +5V source. }
\end {figure}
A multimeter was used to read the value of the output current and
voltage. The circuit was then connected to a 1MHz sine wave and the
input and output was observed using an oscilloscope.Furthermore, a
second voltage divider was the added by connecting a resistor to the
output of the first voltage divider. The value for this resistor was
calculated to be 18,000Ω.
\begin {figure}[!h]
\centering
\includegraphics[scale = 1.0]{double_resistive_voltage_divider.png}
\caption{\label{rvd} A second voltage divider added to the output of
another voltage divider. }
\end {figure} 

\subsection {Capacitor}
A capacitor is an electrical circuit component used to store charge. The
equation I=C(dV/dt) expresses the behavior of a capacitor. The impedance
of a capacitor,Zc, is equal to 1/(jwC) where w is the angular
frequency. Capacitors arranged in series have an equivalent capacitance
of 1/Ceq = 1/C1 + 1/C2 +...+1/C3. Capacitors arranged in parallel have
an equivalent capacitance of Ceq = C1 + C2 +...+1/C3. 

A voltage divider can be built by connecting two capacitors in
series. We built a voltage divider with two capacitors each with value
1μF and connected to a 1MHz sine signal(Figure3).
\includegraphics[scale = 1.0]{capacitive_voltage_divider.png}
\caption{\label{rvd} A capacitive voltage divider with two 1μF
capacitors. }
The input and output signal was observed using an oscilloscope and the
voltage was measured at the output.

We expanded this circuit by adding a resistor to the output. In this
case we used a 1200Ω resistor will be used(Figure4).
\includegraphics[scale = 1.0]{capacitive_voltage_divider_with_resistor.png}
\caption{\label{rvd} A capacitive voltage divider with two 1μF                                
capacitors connected to a 1200Ω load resistor. }
The input and output signal was observed using an oscilloscope. 

\subsection {RC Filters}
Resistors and capacitors can be arranged in series and result in a
filter circuit that allows certain frequencies to enter and rejects
others. There are two types of RC filters, low-pass filter and high-pass
filter. 

\subsubsection {Low-Pass Filter}
A low pass filter is a filter that passes low-frequency signals and
rejects signals with fequencies higher than the cutoff frequency. The
cutoff frequency is the frequency at which the energy of the filter
begins to reduce. The cutoff frequency is expressed as $f = 1/(2\pi CR)$.

We designed a low-pass filter with a cutoff frequency of 106kHz. We
chose a 1600Ω resistor and the value of the capacitor was calculated to
be 10nF. The response of the filter to different frequencies was
observed using an oscilliscope.
 

\subsubsection {High-Pass Filter}
A high-pass filter is a filter that passes high frequency signals and
rejects signals with frequencies lower than the cutoff frequency. We
designed a high-pass filter using the same values for R and C as in the
low-pass filter above and observed the response to different frequencies
using an oscilliscope.

\subsection {LC Circuit}
An LC circuit involves two electrical components, an inductor and a
capacitor. An inductor is a coil that resists the current passing
through it. When current flows through an inductor, energy is stored
temporarily in a magnetic field in the coil. When the current through
the inductor changes, the time-varying magnetic field induces a voltage
in the inductor.An inductor is characterized by its inductance, L, which
is expressed in units of Henries(H). The behavior of an inductor is
expressed as  
\begin{equation}
V = L(dI/dt)
\end{equation}

The impedance of an inductor,Zl, is jwL.

We put together an inductor and a capacitor in series and parallel. The
values for the components used are L = $1\mu H$ and C = $1\mu F$. For these values
the natural frequency, 
$w_o = 1/[LC)^(1/2)]$, is calculated to be 1MHz. The impedance as a
function of angular frequency for the series case then is Z(w) =
$(10^-6)((f^2-10^12)/f)$. The impedance as function of angular frequency
for the parallel case is $Z(w) = (-10^6)(w/(w^2-10^{12}))$. A plot for
impedance vs frequency for both cases was created to further understand
impedance behavior. 

\subsubsection {RLC Filter}

A filter can be built using an LC circuit but a load is required to slow
down the flow of current through the circuit. A resistor is perfect for
this job. Therefore a filter can be built using an RLC circuit. This kind
of filter is called a bandpass RLC filter. A bandpass filter lets in a
set range of frequencies and rejects others. It has a cutoff frequency,
f, that is given by: $f = 1/(2\pi LC)$ and a bandwidth, B, given by: $B =
R/(2\pi L)$. The bandwidth is the difference between the range of
frequencies allowed to pass the filter. 

We designed an RLC bandpass filter with a bandwidth of 200,000Hz. Using values of  
L = $1\mu H$, C = $1\mu F$, the calculated resistance is $1.25\Omega$. A 2.7$\Omega$ resistor
was used instead due to lack of equipment. 

\subsection {Diode}
A diode is a two-terminal circuit components with a low resistance to
current flow in one direction and high resistance in the other. This
characteristic gives the diode the ability to block current in one
direction and allow current to flow in the other. This behavior is
called rectification and is used to convert alternating current to
direct current, which is useful in the extraction of modulation from
radio signals in radio receivers.

We measured the voltage drop across a diode using a multimeter. A
resistor needs to be connected to the diode to limit the current flow
into the diode. Different values of resistors were used and voltages
between 1 and 5V were applied.

\subsection {FM Demodulation}
Demodulation is the process of extracting the original signal from a
modulated incoming signal. A demodulator is an electronic circuit used
to recover the information from the modulated signal. An FM demodulator
takes in a modulated signal in the FM frequency range, which is 88 to
108MHz, and smooths it out into a clean FM signal.





\end{document}