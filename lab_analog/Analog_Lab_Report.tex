\documentclass[12pt]{article}
\usepackage{graphicx}
\usepackage[margin=1in]{geometry}



\begin{document}

\title {Design and Construction of an FM Radio Receiver}
\author {Marta Pastore}
\date {February 10, 2014}
\maketitle 

\begin {abstract}
The goal of this lab is to understand the basic circuit components to
build a foundation for designing an FM radio receiver. To reach this
goal, we began by designing an FM demodulation circuit which was
connected to a radio signal and the output was heard through a
speaker. The design of the FM demodulation circuit included four main
components: a bandpass filter, a bias circuit, a diode/detector, and a
low-pass filter. To improve the performance of our radio receiver, we
designed an amplifier circuit that included five main components, a
high-pass filter, a biasing circuit, a low pass filter, a transistor,
and load resistor. In addition to the amplifier, an emitter follower
circuit, another kind of amplifier, was designed to further improve the
performance of our FM receiver. The FM demodulation circuit was then
connected to both amplifiers and to a radio station. This resulted in a
much improved output signal through the speaker.
\end {abstract}

\section {Introduction}
In this lab we explored the concepts of filters by understanding the
design and behavior of low-pass filters, high-pass filters, and bandpass
filters. Filters are designed around a desired cutoff frequency, which
determines the values of the components of the filter. A low-pass filter
accepts low frequencies and rejects frequencies higher than the cutoff
frequency. A high-pass filter accepts high frequencies and rejects
frequencies lower than the cutoff frequency. A bandpass filter lets in a
set range of frequencies.
 
In addition, the workings of a transistor were explored. We used three
terminal transistors to design more complicated circuits including an
emitter follower circuit and an amplifier circuit. Since both of these
circuits are designed to amplify a signal, the concept of gain was also
discussed. The gain is simply multiplication factor that relates the
magnitude of the output signal to the input signal. The gain can be
expressed as the ratio of output voltage to input voltage (voltage gain)
or output power to input power (power gain).

Lastly, biasing circuits were also used in our more complicated circuit
designs. In our case, a resistive voltage divider was used for
biasing. A biasing circuit simply sets a specific voltage output. We
needed biasing in order for the transistor to behave properly in the
design of the follower and the amplifier.

\section {Methods}

\subsection {Impedance}

Before the main components of a circuit can be explored, the concept of
impedance,Z, needs to be well understood. Electrical impedance is
defined as the amount of opposition a circuit applies to any incoming
current when a voltage is applied. Impedance is therefore analogous to
friction in more physical terms. All of the different components in a
circuit have an impedance and therefore contribute to the overall
opposition of current in the circuit.

\subsection {Resistor}

A resistor is a two-terminal electrical component through which current
can flow. The impedance of a resistor, Zr, is equal to its resistance
R. Ohm's Law, 
\begin {equation}
V = IR
\end {equation}
is a very useful equation in understanding how a
resistor behaves. Furthermore, Resistors can be connected in series to
give a total resistance of
\begin {equation}
Re = R1 + R2 +...+ Rn
\end {equation} 
In this configuration
the voltage across the resistors is the same but the current is
different. Similarly resistor can be connected in parallel to give a
total resistance of 1/Re = 1/R1 + 1/R2 +...+ 1/Rn. In this configuration
the current through the resistors is the same but the voltage is
different.

Resistors can be used to build a voltage divider, which is a circuit
that gives an output voltage that is a fraction of the input
voltage. Two resistors connected in series can be considered a voltage
divider. To test this we took two 1500$\Omega$ resistors and connected them in
series and applied a 5V DC voltage source(Figure 1).

\begin {figure}[h!]
\centering 
\includegraphics[scale = 1.0]{figure1.png}
\caption{\label{rvd} A resistive voltage divider of two 1500$\Omega$ 
resistors connectect two a +5V source. }
\end {figure}

A multimeter was used to read the value of the output current and
voltage. The circuit was then connected to a 1MHz sine wave and the
input and output was observed using an oscilloscope.Furthermore, a
second voltage divider was the added by connecting a resistor to the
output of the first voltage divider. The value for this resistor was
calculated to be 18,000$\Omega$ (Figure 2).

\begin {figure}[h!]
\centering
\includegraphics[scale = 1.0]{figure2.png}
\caption{\label{rvd} A second voltage divider added to the output of
another voltage divider. }
\end {figure} 

\subsection {Capacitor}
A capacitor is an electrical circuit component used to store charge. The
equation I=C(dV/dt) expresses the behavior of a capacitor. The impedance
of a capacitor,Zc, is equal to 1/(jwC) where w is the angular
frequency. Capacitors arranged in series have an equivalent capacitance
of 1/Ceq = 1/C1 + 1/C2 +...+1/C3. Capacitors arranged in parallel have
an equivalent capacitance of Ceq = C1 + C2 +...+1/C3. 

A voltage divider can be built by connecting two capacitors in
series. We built a voltage divider with two capacitors each with value
1$\mu$F and connected to a 1MHz sine signal (Figure 3).

\begin {figure}[h!]
\centering
\includegraphics[scale=1.0]{figure3.png}
\caption{\label{rvd} A capacitive voltage divider of two 1$\mu$F
capacitors. }
\end {figure}

The input and output signal was observed using an oscilloscope and the
voltage was measured at the output.Furthermore, we expanded this circuit
by adding a resistor to the output. In this case we used a 1200$\Omega$
resistor will be used (Figure 4).

\begin {figure}[h!]
\centering
\includegraphics[scale = 1.0]{figure4.png}
\caption{\label{rvd} A capacitive voltage divider of two 1$\mu$F                                
capacitors connected to a 1200$\Omega$ load resistor. }
\end {figure}

The input and output signal was observed using an oscilloscope. 

\subsection {RC Filters}
Resistors and capacitors can be arranged in series and result in a
filter circuit that allows certain frequencies to enter and rejects
others. There are two types of RC filters, low-pass filter and high pass
filter. 
 
A low pass filter is a filter that passes low-frequency signals and
rejects signals with frequencies higher than the cutoff frequency. The
cutoff frequency is the frequency at which the energy of the filter
begins to reduce. The cutoff frequency can be expressed as $f = 1/(2\pi
CR)$. We designed a low-pass filter with a cutoff frequency of
106kHz. We chose a 1600$\Omega$ resistor and the value of the capacitor was
calculated to be 10nF (Figure 5). The response of the filter to
different frequencies was observed using an oscilloscope.

\begin {figure}[!h]
\centering
\includegraphics[scale = 1.0]{figure5.png}
\caption{\label{rvd} The designed low-pass filter circuit diagram. }
\end {figure}
 
A high-pass filter is a filter that passes high frequency signals and
rejects signals with frequencies lower than the cutoff frequency. We
designed a high-pass filter using the same values for R and C as in the
low-pass filter above (Figure 6).We observed the response to different
frequencies using an oscilloscope.

\begin {figure}[!h]
\centering
\includegraphics[scale = 1.0]{figure7.png}
\caption{\label{rvd} The designed high-pass filter circuit diagram. }
\end {figure}

\subsection {LC Circuit}
An LC circuit involves two electrical components, an inductor and a
capacitor. An inductor is a coil that resists the current passing
through it. When current flows through an inductor, energy is stored
temporarily in a magnetic field in the coil. When the current through
the inductor changes, the time-varying magnetic field induces a voltage
in the inductor. An inductor is characterized by its inductance, L,
which is expressed in units of Henries(H).  The behavior of an inductor
is expressed as 

\begin{equation}
V = L(dI/dt)
\end{equation} 

and its impedance,Zl, as jwL.

We put together an inductor and a capacitor in series and parallel
(Figure 7 and Figure 8).

\begin {figure}[!h]
\centering
\includegraphics[scale = 1.0]{figure6.png}
\caption{\label{rvd} A series LC circuit. }
\end {figure}

\begin {figure}[!h]
\centering
\includegraphics[scale = 1.0]{figure8.png}
\caption{\label{rvd} A parallel LC circuit. }
\end {figure}

For the values L = $1\mu H$ and C = $1\mu F$ used, the natural frequency, 
$w_o = 1/[LC)^(1/2)]$, is calculated to be 1MHz. The impedance as a
function of angular frequency for the series case then is Z(w) =
$(10^-6)((f^2-10^12)/f)$. The impedance as function of angular frequency
for the parallel case is $Z(w) = (-10^6)(w/(w^2-10^{12}))$. A plot for
impedance vs frequency for both cases was created to further understand
impedance behavior. 

We designed an RLC bandpass filter with a bandwidth of 200,000Hz (Figure
9). Using values of  L = $1\mu H$ and C = $1\mu F$, the calculated resistance is
1.25$\Omega$. A 2.7$\Omega$ resistor was used instead due to lack of
equipment. 

\begin {figure}[!h]
\centering
\includegraphics[scale = 1.0]{figure9.png}
\caption{\label{rvd} The designed RLC bandpass filter. }
\end {figure}

\subsection {Diode}
A diode is a two-terminal circuit components with a low resistance to
current flow in one direction and high resistance in the other. This
characteristic gives the diode the ability to block current in one
direction and allow current to flow in the other. This behavior is
called rectification and is used to convert alternating current to
direct current, which is useful in the extraction of modulation from
radio signals in radio receivers.

To examine a diode more closely, we measured the voltage drop across a
diode using a resistor to limit the current flow (Figure 10).

\begin {figure}[!h]
\centering
\includegraphics[scale = 1.0]{figure10.png}
\caption{\label{rvd} A diode connected to a resistor.}
\end {figure}

The resistor R was varied and  for each resistor value, the voltage was
varied between 1V and 5V. Next, we used a large 1V oscillating signal
into the diode and viewed the original and rectified signal on an
oscilloscope. Furthermore, we connected to the diode a low pass filter
with time constant of order the input frequency, which in out case was
100kHz. The time constant is therefore 100ms. The values for R and C
were therefore chosen to be 10$\Omega$ and 1$\mu$F, respectively (Figure
10). The output of this circuit was then observed in an oscilloscope.

\begin {figure}[!h]
\centering
\includegraphics[scale = 1.0]{figure11.png}
\caption{\label{rvd} A diode connected to a low-pass filter.}
\end {figure}

\subsection {FM Demodulation}
Demodulation is the process of extracting the original signal from a
modulated incoming signal. A demodulator is an electronic circuit used
to recover the information from the modulated signal. An FM demodulator
takes in a modulated signal in the FM frequency range, which is 88 to
108MHz, and smooths it out into a clean FM signal. 

We built a demodulation circuit using all the different circuits
previously designed. The demodulation circuit was then connected to an
antenna/radio from the input and connected to a speaker on the
output. The station used is 1.045MHz,  which is a fake FM station used
because of the lack of radio reception in the lab. The circuit diagram
for this FM receiver is shown in Figure 12.

\begin {figure}[!h]
\centering
\includegraphics[scale = 0.5]{figure12.jpg}
\caption{\label{rvd} FM demodulation circuit diagram with labeled
unit components.}
\end {figure}

In Figure 12 blocks of components have been highlighted to show the different
parts of the demodulator. Here, the bandpass filter design earlier was
used, giving values of L = 1$\mu$H, C = 1$\mu$F and R =
2.7$\Omega$. The capacitor C1 was given a value of 1$\mu$F. The
resistors, R1 and R2, were both 10$\Omega$ resistors. The Vcc voltage
connected to R1 was set to 5V. The low pass filter was designed to have
a cutoff frequency of 100kHz, resulting in values of 16$\Omega$ for R3
and 0.1$\mu$F for C2. Lastly, a 0.1$\mu$F capacitor was used for C3.

\subsection {Transmission Line}
A transmission line delivers power from a source to a load. It consists
of two conductors and between them there is capacitance. Because any
wire has inductance, a transmission line acts like a series of two
inductors connected in parallel by a capacitor circuits. The
characteristic impedance of a transmission line is as follows
$Z=(L/C)^(1/2)$.

For our experiment we applied a 1MHz square wave function through a 15
meter transmission line. The output was observed using an
oscilloscope. The width of the observed signal was measured using the
divisions on the oscilloscope. A 10Ω resistor was attached to the end of
the transmission and the output was once again observed. Different
resistor values were then tested until for some value the output signal
was a square wave.

\subsection {Transistor}
A transistor is a device used to amplify electronic signals and
electrical power. It is composed of semiconductor material with at least
three terminals for connection to an external circuit. For this a lab, a
three terminal transistor was used. The three terminals are refereed as
the collector, the base and the emitter.
 

Transistors can be used in building an emitter follower. An emitter
follower can serve as a buffer for a voltage source. The voltage gain of
an emitter follower is a little less than one since the emitter voltage
is constrained at 0.6V below the base. Its function is current and power
gain and impedance matching. Its input impedance is much higher than its
output impedance. The low output impedance of the emitter follower
matches a low impedance load and buffers the signal source from that low
impedance. Keeping these concepts in mind, we selected resistor and
capacitor values to build the emitter follower circuit shown in Figure
13.

\begin {figure}[!h]
\centering
\includegraphics[scale = 1]{figure13.png}
\caption{\label{rvd} The designed emitter follower circuit diagram.}
\end {figure}

We connected this circuit to a 10kHz sine wave with amplitude of 1V. The
input and output voltages were observed using an oscilloscope. The
frequency was then increased in order to determine the maximum sine
amplitude for which this circuit operates properly. The bias voltage at
the base was measure and the emitter resister value was decreased until
a change in the bias voltage was observed.
 

Transistors can also be used in building an amplifier. An amplifier is
an electronic device used to increase the power of a signal.  It does
this by taking energy from a power supply and controlling the output to
match the input signal shape but with a larger amplitude. Amplifiers
exhibit the property of gain, or multiplication factor that related the
magnitude of the output signal to the input signal. The gain may be
specified as the ratio of output voltage to input voltage, which is
called voltage gain.

For this lab, we designed the an amplifier represented by the circuit
diagram in Figure 14.

\begin {figure}[!h]
\centering
\includegraphics[scale = 1]{figure14.png}
\caption{\label{rvd} the designed amplifier circuit diagram.}
\end {figure}

The values of the resistors and capacitors were chosen with a gain of 5
in mind. The gain, g, is equal to Rc/Re. We chose Rc to equal 1000$\Omega$, so
Re is equal to 200$\Omega$. We chose R1 to equal 120$\Omega$ and R2 is defined as
(R1)/4, so R2 is equal to 30$\Omega$. A 33$\Omega$ resistor was used for R2 instead
due to lack of 30$\Omega$ resistors in the lab. For C1 and C2, 1$\mu$F capacitors
were used. The emitter capacitor, Ce, was omitted for this part. The Vcc
voltage was set to 5V. The circuit was then connected to a 10kHz sine
wave with 100mV amplitude. The relationship between the input and output
voltages was observed using an oscilloscope in order to determine
whether a gain of 5 was reached. The frequency was then increased in
order to determine the maximum amplitude at which this circuit correctly
operates.
 

Next, we added the emitter capacitor into our circuit with a goal of
reaching a gain of 10 at 10kHz. Now Re is set to 100$\Omega$ and the Ce can be
calculated using the cutoff frequency equation for an RC filter. The
calculated value for Ce is 0.2$\mu$F. We then connected the circuit to a
10kHz sine wave with 100mV amplitude. The relationship between the input
and output voltages was observed using an oscilloscope in order to
determine whether a gain of 10 was reached. The gain at 20kHz was then
measured.
 

Finally, we connected the FM demodulator designed earlier to the
amplifier designed and connected this to the emitter follower previously
designed. The input of the FM demodulator was connected to a 1.05MHz
radio station and the output of the emitter follower was connected to a
speaker.

\subsection {Noise Figure of a receiver}
n electronics, noise is a random fluctuation in an electrical
signal. Noise can be produced by different effects. An example of a type
of noise is the Johnson-Nyquist Noise, which is noise generated by the
thermal agitation of the charge carriers inside an electrical conductor
at equilibrium. For a resistor the voltage behaves as follows:
$V^2=4kTBR$, where k is the Boltzmann constant, T is the temperature, B is
the bandwidth, and R is resistance. Therefore in order to reduce the
noise in a resistor we can lower the temperature of the resistor or
reduce the bandwidth through filtering. 

For this experiment we want to measure the noise through a 50$\Omega$. In order
to do this, the noise needs to be amplified. Therefore, we connected 3
amplifiers, two AC1068C amplifiers with a gain of 24.5dB and one AP1309C
with a gain of 12.5dB. Two 60MHz low pass filters to insure signal is in
the correct bandwidth. The configuration looks as follows:
2W Power supply--Filter--Filter--AP1039C--AC1068C--AC1068C--50$\Omega$
resistor--15V voltage supply.

\section {Results and Discussion}
\subsection {Resistive voltage divider circuit}
For the resistive voltage divider the circuit the measured output
voltage is 2.40V and the measured current is 0.81$\mu$A. Ideally the output
voltage should be 2.5V and the current 0.83$\mu$A. The values are different
sure to thermal losses and other factors. When the circuit was connected
to the 1MHz sine wave, the input and output curves in the oscilloscope
were close to identical; therefore Vin/Vout is close to unity. This
means that a resistive voltage divider does not distort the incoming
signal.
 

For the circuit with the second voltage divider, the two initial
resistors act as a single 3000$\Omega$ resistor from the perspective of the
second voltage divider. This combined resistance is known as the
Thevenin equivalent resistance and its value is helpful in simplifying
the circuit. Furthermore, the third resistor acts as a load on the
voltage-divider circuit. A load draws power from the circuit changing
the output voltage. If R3>>R2, then the ratio Vout/Vin is practically
unchanged because the current chooses to travel the path with least
resistance.
 

When choosing what values of resistors to use in a voltage divider, the
amount of power supply that will be added needs to be taken into
consideration. For really high power supplies, a high impedance is
required in order for the resistor to handle the high input voltage. For
low power supplies, a low impedance is more appropriate in order for the
current to easily travel through the resistor. 

\subsection {Capacitive voltage divider circuit}
For the capacitive voltage divider circuit both input and output curves
are about the same in the oscilloscope. There is just a slight decrease
in amplitude in the output signal. The output voltage is measured to be
2.02V. Overall, a capacitive voltage divider does not distort the
incoming signal.
 

When the 1200$\Omega$ was added to the circuit, there is a bit more of a
difference between Vin and Vout; therefore the  ratio Vin/Vout is not
one at every point but still very close.

\subsection {RC filter response}
For the low pass filter designed the data obtained is shown in Table 1.

\begin {table}[!h]
\center
\begin {tabular}{|c|c|}
\hline
Frequency     &     Voltage  \\
\hline
20kHz         &     1.6V     \\
\hline
50kHz         &     1.6V     \\
\hline
100kHz        &     1.6V     \\
\hline
500kHz        &     450mV    \\
\end {tabular}
\caption { Data for the response of the designed low-pass filter.}
\label {Table.1}
\end {table}

The plot for the response of the low pass filter is shown in Figure 15.
\begin {figure}[!h]
\centering
\includegraphics[scale = 0.5]{figure15.jpg}
\caption{\label{rvd} A plot for the response of the designed low-pass filter.}
\end {figure}

From this plot it can be seen that the voltage begins to gradually
decrease around the cutoff frequency 100kHz.

For the high-pass filter designed, the data obtained is shown in Table
2.
\begin {table}[!h]
\center
\begin {tabular}{|c|c|}
\hline
Frequency     &     Voltage  \\
\hline
20kHz         &     ~0V      \\
\hline
100kHz        &     2V     \\
\hline
500kHz        &     1.8V    \\
\end {tabular}
\caption { Data for the response of the designed high-pass filter.}
\label {Table.2}
\end {table}
The plot for the response of the high-pass filter us shown in Figure 16.
\begin {figure}[!h]
\centering
\includegraphics[scale = 0.5]{figure16.jpg}
\caption{\label{rvd} A plot for the response of the designed high-pass filter.}
\end {figure}

From this plot it can be seen that the voltage gradually increases up to
the cutoff frequency of 100kHz and the stays there for higher
frequencies. 

To make the lines of the filter steeper, better capacitors and resistors
can be used to improve performance of the filter.

\subsection {LC circuit and bandpass filter response}
For the series LC circuit designed earlier the plot of impedance vs
frequency is shown in Figure 17.

\begin {figure}[!h]
\centering
\includegraphics[scale = 0.5]{figure17.jpg}
\caption{\label{rvd} A plot of impedance vs frequency for the designed
series LC circuit.}
\end {figure}

From this plot we can see that as the applied frequency approaches the
natural frequency of the circuit, 100kHz, the impedance approached
zero. This means that at frequencies very high from the natural
frequency, the series LC circuit behaves a a regular a wire.

For the parallel LC circuit designed earlier the plot of impedance vs
frequency is shown in Figure 18.

\begin {figure}[!h]
\centering
\includegraphics[scale = 0.5]{figure18.jpg}
\caption{\label{rvd} A plot of impedance vs frequency for the designed
parallel LC circuit.}
\end {figure}

\subsection {Diode with resistor and low-pass filter}
The data for the voltage across a diode for various resistors and
voltages is the following:

When the diode was connected to the low pass filter, across the resistor
the signal was noisy and across the diode the signal was
smooth. Therefore the diode took in a high frequency signal and provided
an output that is an envelope of the original signal, giving a more
spread out curve on the oscilloscope. 

\subsection {Speed of signal and characteristic impedance of a
transmission line}
When we applied a 1MHz square wave to out 15m transmission line and
connected the output to an oscilloscope, the observed graph is shown in
Figure 19.

\begin {figure}[!h]
\centering
\includegraphics[scale = 0.5]{figure19.jpg}
\caption{\label{rvd} A replication of the output in an oscilliscope for
a 1MHz square wave applied to a 15m transmission line.}
\end {figure}

Here, the reflected wave causes a step on the square wave. The width of
this plot was measured to be 750ns. This values is the amount of time it
took the signal to travel to en of the line and back. Therefore, the
speed of the signal is $4x10^(7)m/s$. When we added the 10$\Omega$
resistor to terminate the far end of the cable and kept the same input
signal, the observed graph is shown in Figure 20.

\begin {figure}[!h]
\centering
\includegraphics[scale = 0.5]{figure20.jpg}
\caption{\label{rvd} A replication of the output in an oscilliscope for
a 1MHz square wave applied to a 15m transmission line terminated with a
10$\Omega$ resistor.}
\end {figure}

After trying various resistor values in order to properly terminate the
signal and get a square wave on the oscilloscope, a 56$\Omega$ resistor was
found to properly terminate the signal. The signal is properly
terminated when the resistor is equal to the characteristic impedance of
the cable; therefore the 15m transmission used has a characteristic
impedance of 56$\Omega$. 

\subsection {Emitter follower input and output observations}

For the emitter follower design, the measured bias voltage was 0.6V. The
output signal observed in the oscilloscope was smoother than the input
signal. The maximum frequency at which this circuit operates was found
to be 600kHz. At this frequency, the output signal is not longer
smooth and the signal is almost lost. 

When the bias voltage at the base was measured with a 270k$\Omega$
resistor as the emitter resistor, the bias voltage at the base was
measured to be 7.78V. When we changed the emitter resistor value to
150k$\Omega$ resistor, the voltage dropped to 7.6V. Therefore, this is
the minimum frequency at which the voltage at the base will stay
constant.

\subsection {Amplifier input, output and gain observations}
When the designed amplifier circuit was connected to a 10kHz sine wave
with 100mV amplitude,the voltage at the collector was measure to be
3.87V. When the output signal was seen in the oscilloscope, there was an
amplification of the signal with a gain of about 4. The gain of 5 was not
reached due to the values chosen for the capacitors and resistors in the
circuit. The maximum frequency at which this circuit properly operates
was found to be 500kHz. 

When we considered a gain of ten and added the emitter capacitor into
the circuit, the output signal was still amplified in the
oscilloscope. A gain of about 7 was reached. The gain of 10 was not
reached due to the values used for the capacitors and resistors. Further
calculation of these values will result in a gain closer to 10. 

Finally when the FM demodulator, the amplifier and the follower designed
were combined and connected to a 1.04MHz radio and signal and a speaker,
we were about to listen to the radio station, which means out circuits
worked properly to transmit the signal. 

\subsection {Detecting the noise figure of a receiver}
The experiment for this part was set as 2W Power
supply--Filter--Filter--AP1039C--AC1068C--AC1068C--50Ω resistor--15V
voltage supply. Unfortunately, only one of the amplifiers in the lab
was working properly. Without the amplifiers the noise cannot be detected
properly. Therefore we were unable to conduct the experiment.

\section {Conclusion}
The FM radio receiver designed and built in this lab incorporated
various concepts and allowed us to become familiar with circuit
elements. The measuring of the noise through a receiver is something of
interest to complete in the future. In addition, exploring the design of
other kinds of receivers is something we plan to do in order to further
understand electronics and all of their components.

\section {Acknowledgments}
I would like to thank Eduardo Herrera who contributed to the results of
this lab. I would also like to thank Baylee Bordwell and Karto Keating
for providing guidance when needed. 

\end{document}